\RequirePackage{plautopatch}
\RequirePackage[l2tabu, orthodox]{nag}

\documentclass[platex,dvipdfmx]{jlreq}			% for platex
% \documentclass[uplatex,dvipdfmx]{jlreq}		% for uplatex
\usepackage{graphicx}
\title{数式テスト}

\author{岩瀬 憲一}
\date{\today}
\begin{document}
\maketitle
\section{数式の書き方}
式$x^2 + y^2$はディスプレイ数式で
\[
x^2 + y^2
\]
と表される。\\
また、正弦関数の加法定理は次のようになる。
\begin{equation}
  \sin(x+y) = \sin x \cos y + \cos x \sin y
\end{equation}
\section{分数の表現}
式$\frac{1}{a} + \frac{1}{b} = \frac{a+b}{ab}$はディスプレイ数式で
\[
\frac{1}{a} + \frac{1}{b} = \frac{a+b}{ab}
\]
と表される。
\section{添字の表現}
総和記号$\sum_{n=1}^\infty$をディスプレイ数式で書くと、
\[
\sum_{n=1}^\infty
\]
となる。\\
積分記号$ \int_a^b f(x)dx $をディスプレイ数式で書くと、
\[
\int_a^b f(x)dx
\]
となる。
\end{document}
